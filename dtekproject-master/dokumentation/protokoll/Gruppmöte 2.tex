\documentclass[a4paper]{article}

\usepackage[swedish]{babel}
\usepackage[T1]{fontenc}
\usepackage[utf8]{inputenc}
\usepackage{graphicx}
\usepackage{makecell}
\setlength\parindent{0pt} %changes indentation size to 0

\begin{document}

\title{Gruppmöte 2}
\date{2020-09-02}
\maketitle

\section{Mötets öppnande}
\label{sec:roller}
Närvarande
Felix Bråberg
Linus Haraldsson
Erik Nilsson
Simon Widerberg
Philip Antonsson
Gabriel Käll
Sebastian Sjögren

\section{Val av sekreterare och justerare}
\label{sec:val}
Sekreterare: Erik Nilsson
\newline
Justerare:  Philip Antonsson


\section{Godkännande av dagordning}
\label{sec:godk}
Tillägg av punkter






\section{Godkännande av föregående mötesprotokoll}
\label{sec:godk}






\section{Informationspunkter}
\label{sec:inf}
Träffas idag 09/09 kl. 13:00 för att arbeta med projektet
Fredags övningspass: Philip, Linus, Simon, Erik. Efteråt förklara för de andra.




\section{Statusrapport}
\label{sec:stat}
Vi har jobbat med projektplanen och förväntas bli klara i veckan.
Vi har totalt jobbat cirka 15 timmar denna vecka inviduellt.


\newline \newline


Gabriel: Jobbat med bakgrunden, skrivit mycket på tekniska förutsättningar, även beskrivit ett allmänt larmsystem. Ska jobba mer på den och lägga till mer referenser. Börjat på ett kodkonventionsdokument separat från planeringen.

\newline \newline



Philip: Gjorde en google kalender och i projektplanen jobbade med mötesplan och tidsplan, är klar med mötesplan texten, gantschemat är också klart. Är redo att hjälpa andra med deras.

\newline \newline



Felix: Jobbat med Målet och syftet, sedan hjälpte till med milstolpar-delen.


\newline \newline


Sebastian: Jobbat med kommunikationsplanen och spelregler och är klar med dem.

\newline \newline



Simon Wider:  Jobbat med milstolpar, verkar rätt fullständingt och lagt in en tabell.

\newline \newline

Linus: Jobbat med kvalitetsplanen, är inte klar än men har kommit en bit och förväntar vara klar tills imorgon.

\newline \newline


Erik: Jobbat med projektplanen, är i stort sätt klar men behöver skriva in mer tilldelningar av arbetsuppgifter när vi delat ut dem m.m.




\section{Uppföljning och eventuella förändringar}
\label{sec:uppf}
Finjustering och mindre förändringar med dokumentet.





\section{Kommande uppgifter}
\label{sec:komm}
Få in projektplanen och gå över kodkonventionerna samt göra flödesschema och dela in oss i mindre grupper och utdela uppgifter till var och en. Övningspass på fredag sedan är det läge att börja med själva uppgiften, försöka få en känsla för hårdvaran. 
Möte på måndag inskrivet i kalendern.
Titta på bedömningskriterierna och göra eventuella ändringar.



\section{Övriga frågor}
\label{sec:övr}
Hur detaljerat ska milstolpar beskrivas?
	Rätt övergripande men det vi har borde vara bra.


I Bakgrunden hur ingående bör man förklara CAN och hur den funkar?
	Inte i detalj men man kan nämna den rätt övergripande.

Diskussion om branches


\section{Nästa möte sker}
\label{sec:övr}
Onsdag 17/09 kl. 14:00 zoom


\section{Mötets avslutande}
\label{sec:övr}



Vid protokollet

Sekreterarens namn
Erik Nilsson
-----------------------------------------



Justeras


Justerarens namn
Philip Antonsson
-----------------------------------------


\end{document}






