\documentclass{article} 
\usepackage[utf8]{inputenc} 
\title{Gruppmöte 7} 
\date{14-10-2020, kl.15:00, Över Zoom}
 \begin{document} \maketitle

\section{Mötets öppnande} 
Närvarande
Felix Bråberg
Linus Haraldsson
Erik Nilsson
Simon Widerberg
Philip Antonsson
Gabriel Käll
Sebastian Sjögren

\section{Val av sekreterare och justerare} 
Sekreterare: Erik Nilsson
\newline
Justerare: Gabriel Käll

\section{Godkännande av dagordning}

\section{Godkännande av föregående mötesprotokoll} 

\section{Informationspunkter} 
Static gör att variablerna/funktionerna inte går att komma åt vid include.

\section{Statusrapport}
Felix 21tim: Gjort oppositionen, skrivit testmallar och jobbat med rapporten.
\newline\newline
Gabriel 23,5: Skrev om tekniska förutsättningar och bakgrunden, förbättrat beskrivningen av centralenheten och jobbat med opposition.
\newline\newline
Philip 20,5tim: Jobbat med texten; syfte arbetsmetod. Testat dörrenheten och  jobbat med koden. Även gjort oppositionen.
\newline\newline
Sebastian 23: Testat och skrivit kod för centralenhetens IO samt jobbat med rapporten och oppositionen.
\newline\newline
Simon 23tim: Opponering, skrivit om översikten i rapporten, har en del rapportskrivande kvar, testat hårdvaran och suttit med ljud.
\newline\newline
Linus 20tim: Gjort klart typ hela nätversklagret för larmenheten, städat kod, testat och skrivit rapport.
\newline\newline
Erik 21tim: Opponering, skrivit rapport, i stort sett blivit klar med dörrenhetskoden, testat den för sig men inte med alla andra enheter samtidigt.


\section{Uppföljning och eventuella förändringar} 

\section{Kommande uppgifter}
Rapportutkast på söndag; vi ska korsexaminera rapporten och redigera den för att kunna lämna in den, utkastet borde komma tillbaka tidigt nästa vecka. Vi får en teknisk återkoppling samt skriftmässigt.
\newline\newline
Demon är helt på distans och en video ska spelas och skickas in. Max 20 min.
Kort efter det ska sista rapportinlämningen in.
\newline\newline
Testa hela systemet.
\newline\newline
Göra störenheten, går bra att använda strukten för att göra den.
\newline\newline
Diskutera hur vi ska disponera avsnitt tre och fyra, vidareutveklingsmöjligheter; på helhetnivå.
\newline\newline
Grupparbetsutvärdering: reflektion över projektarbetet, allmänt hur arbetet har gått, koppla till projektplanen. Skrivs tillsammans på torsdag.
\newline\newline
1:a november enskild grupparbetsutvärdering.
\newline\newline
Städa upp i koden och skriva kommentarer enligt kodkonventionerna.
\newline\newline
Merga alla branches i GitHub, då behöver en del kod flyttas runt exempelvis för CAN.  CAN.h har de variablerna som behövs kunna kommas åt.


\section{Övriga frågor} 

\section{Nästa möte sker...} 

kl 15.00 21/10

\section{Mötets avslutande} 

\end{document}
