\documentclass{article} 
\usepackage[utf8]{inputenc} 
\title{Gruppmöte 7} 
\date{14-10-2020, kl.15:00, Över Zoom}
 \begin{document} \maketitle

\section{Mötets öppnande} 
Närvarande
Felix Bråberg
Linus Haraldsson
Erik Nilsson
Simon Widerberg
Philip Antonsson
Gabriel Käll
Sebastian Sjögren

\section{Val av sekreterare och justerare} 
Sekreterare: Erik Nilsson
\newline
Justerare: Philip Antonsson

\section{Godkännande av dagordning}

\section{Godkännande av föregående mötesprotokoll} 

\section{Informationspunkter} 
Vi ligger bra till enligt handledare.
Måndag kl. 12:30 i basen, hårdvaruansvarig lämnar hårdvaran.
Bild till presentationen.


\section{Statusrapport}
Felix 21tim: Skrivit rapport inför utkastet, gjort tester på  dörr och larmenhet, skrivit dokumentation för funktioner.
\newline
Gabriel 27tim: Hälften rapport hälften på plats med tester och kod.
\newline
Philip 24 tim 40 min: Skrivit på rapporten, städande av dörrkoden, klar med dörrenheten. Tester.
\newline
Sebastian 26 tim: Projektrapportskrivande, fixa bugs, tester.
\newline
Simon Widerberg 22tim: Rapport, examinering, tester, infört en kalibreringsfunktion för avståndssensorn.
\newline
Linus 30 tim: Mycket rapportskrivande, kalibrering, städat, tester, störenheten.
\newline
Erik 21tim: Rapportskrivning, städat kod, tester, dörrenheten är klar vad vi ser.


\section{Uppföljning och eventuella förändringar} 

\section{Kommande uppgifter}
Testa hela systemet, merga, demoinspelning(20min ung.)
\newline
Städa kod.
\newline
Manus till demon.
\newline
Presentationstekniskt hjälpmedel.


\section{Övriga frågor} 
Ang. avståndssensorn - Motivera 2 cm precision med avståndssensorn
\newline
Ang. Dörrsensorer - Använd dörrsensorer. Finns att hämta i lärarrummet.
\newline
Hur ska man visa störenhetens funktion - Visa att den är inkopplad och skickar, visa att dörrsensorn är funktionell samtidigt. Hitta på ett då det funkar och ett extremt.
\newline
Rapportutkastet - Vi kommer få återkoppling senare imorgon och språk på onsdag nästa vecka.
Ska man visa bara styrkor eller inte? -> Eventuellt men inte i mycket detalj men för störenheten är det mer intressant. Mer intressant att visa funktionaliteten i allmänhet.

\section{Nästa möte sker...} 

N/A
\section{Mötets avslutande} 

\end{document}
