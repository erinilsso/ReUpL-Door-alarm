\documentclass{article} 
\usepackage[utf8]{inputenc} 
\title{Gruppmöte 5} 
\date{23-09-2020, kl.14:00, Över Zoom}
 \begin{document} \maketitle
 \section*{Preliminär dagordning:} 

\section{Mötets öppnande} 
Närvarande
Felix Bråberg
Linus Haraldsson
Erik Nilsson(sen)
Simon Widerberg
Philip Antonsson
Gabriel Käll
Sebastian Sjögren

\section{Val av sekreterare och justerare} 

Sekreterare: Sebastian Sjögren, Justerare Felix

\section{Godkännande av dagordning}

\section{Godkännande av föregående mötesprotokoll} 

\section{Informationspunkter} 

\begin{itemize} \end{itemize} 

\section{Statusrapport}

Eri 13h: Jobbade på dörrenheterna, få det att fungera, jobba på interrupt. 
\newline
Seb 20h, Jobba på fungerande kommandon, städa kod, föreläsningar, projektrapport.
\newline
phil, 20h, delay, systick, avbrott för dorr, logik, ta emot meddelande.
\newline
Gab: 20h, CAN jobb, encode/decode, projektrapport.
\newline
Fel: 17h, CAN, intakt struct, hanteringsfunktion för CAN-receive. kollat föreläsning, nätverkslager. 
\newline
Sim: 15h, Avståndsmätare att fungera och kalibrerbar, städa kod, jobba med ljud.
\newline
Lin 14h: Avståndsmätare att fungera och kalibrerbar, städa kod.

\section{Uppföljning och eventuella förändringar} 

Merga allas kod och få det fungera, köra tester, skriva projektrapport och lämna in projektrapport utkast 1.

\section{Kommande uppgifter} 

\section{Övriga frågor} 

Tester, funktionella test på problematiska/viktiga funktioner, prestanda test (mest med störenhet,störenhet inte riktigt gjord tills nästa vecka dock.)

Can, räkna prioritet. vissa meddelanden kommer alltid före andra. Interpret funktion, can\textunderscore init() funktion för can att fungera. 

Störenhet, spamma random nummer, får komma på random funktion. 

Projektrapport struktur. Beskrivande text av programmet i teknisk beskrivning, inte faktiskt kod. 

Oppositionsrapport deadline, när får man sin rapport? Grupp får sin rapport att opponera på måndag femte oktober 10/5 (enligt Rasmus).

Plan till plats arbete på chalmers. Alla åker in 01/10 för att arbeta med hårdvara på plats.

Ljudbibliotek, hur hz/frekvens konfigureras genom biblioteket. Massa siffror i koden speglar en sinuskurva, beräknade siffror. Simon kollar på det själv.

Hur vi börjar merga (Philip): 1. måste ha funktionen receiver 2. köra can\textunderscore init(receiver) 3. använd interpret\textunderscore message 4. använd general.h

\section{Nästa möte sker...} 

kl 15.00-16.00 7/10

\section{Mötets avslutande} 

\end{document}
