\documentclass[a4paper]{article}

\usepackage[swedish]{babel}
\usepackage[T1]{fontenc}
\usepackage[utf8]{inputenc}
\usepackage{graphicx}
\usepackage{makecell}
\setlength\parindent{0pt} %changes indentation size to 0

\begin{document}

\title{Gruppmöte 3}
\date{2020-09-16}
\maketitle

\section{Mötets öppnande}
\label{sec:roller}
Närvarande
Felix Bråberg
Linus Haraldsson
Erik Nilsson
Simon Widerberg
Philip Antonsson
Gabriel Käll
Sebastian Sjögren

\section{Val av sekreterare och justerare}
\label{sec:val}
Sekreterare: Erik Nilsson
\newline
Justerare:  Simon Widerberg


\section{Godkännande av dagordning}
\label{sec:godk}
Tillägg av punkter






\section{Godkännande av föregående mötesprotokoll}
\label{sec:godk}






\section{Informationspunkter}
\label{sec:inf}
Vi kommer få en mer ingående utvärdering av projektplanen kommande veckor.
Dåligt med blockerande funktioner.
Vi har planerat att minimera mängden data som skickas över canbussarna.




\section{Statusrapport}
\label{sec:stat}
Vi blev klara med projektplanen, höll oss till schemat. Läste på om enheterna. Cirka 10 timmar var och en tillsammans
\newline \newline
Vi skrev in delar i projektplanen och jobbade med den. Läste igenom lite mer och letat efter mer info. Gabriel gjorde även ett dokument med kodkonventioner och Linus läste på om C-språket. Vi har även tänkt över hur enheterna skall fungera och medverka med varandra, funderat på strukter. Vi påbörjade projektrapporten.




\section{Uppföljning och eventuella förändringar}
\label{sec:uppf}





\section{Kommande uppgifter}
\label{sec:komm}
Vi kommer att träffas torsdag och fredag för att sätta oss in mer i hårdvaran och börja koda, och skriva in i rapporten parallellt, göra en skiss eventuellt. Gå igenom kodkonventionerna.
Jobba ett pass på fredagen efter hårdvaruarbetet. Skapa en mappstruktur för projektet. Läsa igenom “project management” dokumentet.
\newline \newline

Mål att veta mer om enheterna ex.: hur larmenheten skickar över data, ha lyckats skicka över data mellan enheterna.


\section{Övriga frågor}
\label{sec:övr}
Hur mycket man ska ha skrivit till första projektrapportutkastet
Lägg energin på det som går men kommer få återkoppling med ändringar till den
riktiga inlämningen.

\section{Nästa möte sker}
\label{sec:övr}
Onsdag 23/09 kl. 14:00 zoom

\section{Mötets avslutande}
\label{sec:övr}



Vid protokollet

Sekreterarens namn
Erik Nilsson
-----------------------------------------



Justeras


Justerarens namn
Simon Widerberg
-----------------------------------------


\end{document}
